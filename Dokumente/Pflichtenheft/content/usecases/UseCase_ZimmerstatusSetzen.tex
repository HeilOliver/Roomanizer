%%
%% Author: Oliver Heil
%% 08.03.2018
%%


\documentclass[./detailed_overview_usecases.tex]{subfiles}
\begin{document}

    \subsection{Zimmerstatus setzen}
    \subsubsection{Detaillierte Benutzungsfallbeschreibungen}
    \textit{Primary Actor:}
    \begin{enumerate}
        Front-Office Personal
        Back-Office Personal
        Reinigungsfachkraft
    \end{enumerate}
    \\
    Stakeholder and Interests:
    \begin{itemize}
        \item[-] Front/Back-Office Personal: Möchte wissen ob das Zimmer beim Check-In vom Kunden bezogen werden kann.
        \item[-] Reinigungsfachkraft: Welche Zimmer gereinigt werden können.
    \end{itemize}

    \subsubsection*{Preconditions}
    -
    \subsubsection*{Postconditions}
    Der Zimmerstatus wurde auf einen neuen Status gesetzt, bzw. es wurde noch kein Zimmerstatus festgelegt.

    \subsubsection*{Main Success Scenario}
    \begin{enumerate}
        % ... es heißt wohl immer Status egal ob Singular oder Plural: https://www.duden.de/rechtschreibung/Status
        \item Zimmerstatus kann auf folgende Status geändert werden
        \begin{itemize}
            \item[a.] BESETZT – GEREINIGT
            \item[b.] BESETZT – UNGEREINIGT
            \item[c.] FREI – GEREINIGT
            \item[d.] FREI – UNGEREINIGT
            \item[f.] OUT OF ORDER
        \end{itemize}
        \item Der Mitarbeiter fragt die Information auf welchen Status das Zimmer gesetzt ist ab. Dazu wird die Zimmernummer eingegeben oder aus einer Liste aller Zimmer gewählt.
        \item Das System zeigt den momentanen Status der Zimmer an.
        \item Der Mitarbeiter trägt den gewünschten Status in das System ein.
        \item Das System setzt den neuen Status.
    \end{enumerate}

    \subsubsection*{Extensions}
    -

    \subsubsection{Sequenz Diagramme}
    \subsubsection{Kontrakte}
\end{document}