%%
%% Author: Oliver Heil
%% 08.03.2018
%%


\documentclass[./detailed_overview_usecases.tex]{subfiles}
\begin{document}

    \subsection{Zimmerstatus setzen}
    \subsubsection{Detaillierte Benutzungsfallbeschreibungen}
    \textit{Primary Actor:}
    \begin{enumerate}
        Front-Office Personal
        Back-Office Personal
        Reinigungsfachkraft
    \end{enumerate}
    \\
    Stakeholder and Interests:
    \begin{itemize}
        \item[-] Front/Back-Office Personal: Kann das Zimmer beim Check-In vom Kunden bezogen werden.
        \item[-] Reinigungsfachkraft: Welche Zimmer gereinigt werden können.
    \end{itemize}

    \subsubsection*{Preconditions}
    -
    \subsubsection*{Postconditions}
    Der Zimmerstatus wurde auf einen neuen Status gesetzt.

    \subsubsection*{Main Success Scenario}
    \begin{enumerate}
        \item Zimmerstatus kann auf folgende Statusse geändert werden
        \begin{itemize}
            \item[a.] BESETZT – GEREINIGT
            \item[b.] BESETZT – UNGEREINIGT
            \item[c.] FREI – GEREINIGT
            \item[d.] FREI – UNGEREINIGT
            \item[f.] OUT OF ORDER
        \end{itemize}
    \end{enumerate}

    \subsubsection*{Extensions}
    -

    \subsubsection{Sequenz Diagramme}
    \subsubsection{Kontrakte}
\end{document}