%%
%% Author: Robert
%% 08.03.2018
%%

\documentclass[./detailed_overview_usecases.tex]{subfiles}
\begin{document}

    \subsection{Checkout}
    \subsubsection{Detaillierte Benutzungsfallbeschreibungen}
    \textit{Primary Actor: Front-Office Personal}
    % describe actor here
    \\
    \textit{Stakeholder and Interests:}
    \begin{itemize}
        \item[-] Front-Office Personal: Möchte dem Individualgast eine unkomplizierte Abreise ermöglichen.
        \item[-] Individualgast: Möchte Informationen über seinen Aufenthalt erhalten und anschließend abreisen.
    \end{itemize}

    \subsubsection*{Preconditions}
    Der Individualgast hatte einen Aufenthalt im Hotel und möchte nun Abreisen.

    \subsubsection*{Postconditions}
    %post conditions text
    Die Abreise des Individualgastes wird ermöglicht.

    \subsubsection*{Main Success Scenario}
    \begin{enumerate}
        \item Das Front-Office Personal teilt dem System den Checkout eines Individualgastes mit.
        \item Das System erstellt eine Zwischenrechnung (UseCase_ZwischenrechnungErstellen)
        \item Der Individualgast überprüft die Einzelheiten der Zwischenrechnung.
        \item Das Front-Office Personal teilt dem System mit die Rechnung zu legen.
        \item Das System legt die Rechnung (UseCase_RechnungLegen).
        \item Der Individualgast bezahlt die Rechnung.
        \item Das System erstellt die Rechnung (UseCase_RechnungErstellen).
        \item Das Front-Office Personal händigt die Rechnung an den Individualgast aus.
        \item Der Zimmerstatus wird auf UNGEREINIGT gesetzt (UseCase_ZimmerstatusÄndern).
    \end{enumerate}

    \subsubsection*{Extensions}
    \begin{enumerate}
        \item Verlängerung:
                \begin{itemize}
                       \item[a.] Der Individualgast möchte seinen Aufenthalt verlängern.
                            \begin{itemize}
                                \item[i.] Der Aufenthalt wird verlängert (UseCase_AufenthaltVerlängern)
                            \end{itemize}
                \end{itemize}
        \setcounter{emuni}{2}
        \item Fehler auf der Zwischenrechnung:
            \begin{itemize}
                \item[a.] Dem Individualgast fällt ein Fehler auf der Zwischenrechnung auf.
                \begin{itemize}
                    \item[i.] Der Fehler wird auf der Zwischenrechnung korrigiert.
                    \item[ii.] Es kann aus der Sicht des Personals kein Fehler festgestellt werden.
                    \item[iii.] Zwischenrechnung wird erstellt (UseCase_ZwischenrechnungErstellen).
                    \item[iv.] Der Individualgast überprüft die Zwischenrechnung.
                \end{itemize}
            \end{itemize}
        \setcounter{emuni}{2}
        \item Fehler auf der Zwischenrechnung:
        \begin{itemize}
            \item[a.] Der Individualgast möchte die Rechnung teilen.
            \begin{itemize}
                \item[i.] Die Rechnung wird geteilt (UseCase_RechnungTeilen).
            \end{itemize}
        \end{itemize}
        \item Fehler auf der Endrechnung:
        \begin{itemize}
            \item[a.] Dem Individualgast fällt ein Fehler auf der bereits gelegten Rechnung auf.
            \begin{itemize}
                \item[i.] Die bereits gelegte Rechnung wird storniert (UseCase_RechnungStornieren)
            \end{itemize}
        \end{itemize}
        \setcounter{enumi}{5}
        \item Bezahlung: \begin{itemize}
                             \item[a.] Der Individualgast möchte, dass ihm die Rechnung zugesendet wird.
                             \begin{itemize}
                                 \item[i.] Die Rechnung wird geteilt (UseCase_RechnungTeilen)
                                 \item[i.] Die Rechnung wird gesendet.
                             \end{itemize}
        \end{itemize}
    \end{enumerate}

    \subsubsection{Sequenz Diagramme}
    \subsubsection{Kontrakte}
\end{document}