%%
%% Author: Moritz
%% 14.03.2018
%%

\documentclass[./detailed_overview_usecases.tex]{subfiles}
\begin{document}

    \subsection{Zimmer wechseln}
    \subsubsection{Detaillierte Benutzungsfallbeschreibungen}

    \textit{Primary Actor: Front-Office Personal}
    % describe actor here
    \\
    \textit{Stakeholder and Interests:}
    \begin{itemize}
        \item[-] Front-Office Personal: Möchte bei Zimmerwechsel, dass der Aufenthalt des Gastes geändert wird und keine Inkonsitenzen in Zusammenhang mit der neuen
        Zimmernummer und der Rechnung entstehen. Außerdem sollen Sonderfälle vom System automatisch erkannt und bearbeitet werden können.
        \item[-] Individualgast/Gast: Möchte in ein anderes freies Zimmer mit den gewünschten Zusatzleistungen und evtl. einer anderen Zimmerkategorie wechseln.
    \end{itemize}

    \subsubsection*{Preconditions}
    Ein Check-In ist bereits erfolgt und der Individualgast hat seinen Aufenthalt begonnen. Es muss freie Zimmer im Hotel geben.

    \subsubsection*{Postconditions}
    Der Individualgast kann ein neues Zimmer beziehen, wobei alle offenen Beträge mit übernommen werden und der Gast nun mit
    seiner neuen Zimmernummer im System vermerkt ist. Das ursprüngliche Zimmer ist freigegeben oder weiterhin belegt.

    \subsubsection*{Main Success Scenario}
    \begin{enumerate}
        \item Das Front-Office Personal gibt die aktuelle Zimmernummer des Gastes oder die Belegungsnummer ein.
        \item Das System liefert alle Informationen zum aktuellen Aufenthalt (Belegung) des Gastes zur eingegebenen Zimmernummer.
        \item Das Front-Office Personal nimmt im System eine neue Zimmerzuweisung vor nachdem die zuvor angzeigten Informationen durch den Gast bestätigt wurden.
        \item Das System setzt die Belegungsnummer der aktuellen Rechnung für das ursprüngliche Zimmer auf 1 und liefert eine Auswahl freier Zimmer in den Kategorien.
        \item Das Front-Office Personal wählt ein freies Zimmer in der vom Gast gewünschten Kategorie mit den gewünschten möglichen Zusatzleistungen.
        \item Das Front-Office Personal bestätigt die vom System gezeigte Sicherheitsabfrage.
        \item Das System übernimmt alle offenen Positionen des ursprünglichen Zimmers und verknüpft diese mit der neuen Zimmernummer. Das ursprüngliche Zimmer
        erhält nun den Status FREI - UNGEREINIGT.
    \end{enumerate}

    \subsubsection*{Extensions}
    \begin{enumerate}
        \item Zimmer kann nicht gefunden werden.
            \begin{itemize}
                \item[a.] Der aktuelle Aufenthalt kann über die Zimmernummer nicht gefunden werden.
                    \begin{itemize}
                        \item \item[i.] Der aktuelle Aufenthalt wird über die Belegungsnummer gesucht, insofern diese verfügbar ist.
                    \end{itemize}
                \item[b.] Der aktuelle Aufenthalt kann über die Belegungsnummer nicht gefunden werden.
                       \begin{itemize}
                           \item[i.] Der aktuelle Aufenthalt wird über den Namen des Gastes gesucht.
                        \end{itemize}
            \end{itemize}
        \setcounter{enumi}{2}
        \item Der Individualgast, der das Zimmer wechseln möchte ist nicht im System unter seinem Namen vermerkt.
        \begin{itemize}
            \item[a.] Der Gast der das Zimmer wechseln möchte, hat das Zimmer nicht gebucht.
                \begin{itemize}
                    \item[i.] Die verantwortliche Person (Gruppenleiter, Unternehmen) muss den Wechsel vornehmen.
                \end{itemize}
        \end{itemize}
        \item Das Zimmer bleibt weiterhin belegt mit Mehrfachbelegung
        \begin{itemize}
            \item[b.] Das ursprüngliche Zimmer bleibt weiterhin belegt, allerdings als Mehrfachbelegung
                \begin{itemize}
                        \item[i.] Die Belegungsnummer wird nicht geändert.
                \end{itemize}
        \end{itemize}
        \setcounter{enumi}{4}
        \item Es gibt kein freies Zimmer
            \begin{itemize}
                \item[a.] In jeder Kategorie sind alle verfügbaren Zimmer bis zur Überbuchungsgrenze gebucht.
                    \begin{itemize}
                        \item[i.] Der Zimmerwechsel kann nicht vorgenommen werden. Der UseCase wird beendet.
                    \end{itemize}
                \item[b.] Es gibt keine freien Zimmer mehr, aber Überbuchungen in der Kategorie sind noch möglich
                    \begin{itemize}
                        \item[i.] Das Front-Office Personal nimmt eine Überbuchung vor.
                     \end{itemize}
                \item[c.] Es gibt keine freien Zimmer mehr mit den gewünschten Zusatzleistungen in der Kategorie
                        \begin{itemize}
                            \item[i.] Das Front-Office Personal findet eine Alternative in einer anderen Zimmerkategorie gemeinsam mit dem Individualgast.
                        \end{itemize}
            \end{itemize}
        \setcounter{enumi}{6}
        \item Das ursprüngliche Zimmer bleibt weiterhin belegt.
            \begin{itemize}
                \item[a.] Das ursprüngliche Zimmer bleibt weiterhin belegt (auch für Mehrfachbelegung).
                    \begin{itemize}
                        \item[i.] Der Status des ursprünglichen Zimmers wird nicht geändert.
                    \end{itemize}
            \end{itemize}
    \end{enumerate}

    \subsubsection{Sequenz Diagramme}
    \subsubsection{Kontrakte}

\end{document}