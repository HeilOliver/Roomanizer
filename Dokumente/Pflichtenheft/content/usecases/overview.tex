%%
%% Author: Moritz
%% 18.03.2018
%%

% Preamble
\documentclass[../../Pflichtenheft.tex]{subfiles}
\begin{document}
In den folgenden Kapitel werden verschiedene Benutzungsfälle der Software beleuchtet.
Ein Benutzungsfall stellt die Interaktion zwischen eines Anwenders und dem System dar, um ein gewisses Ziel erreichen zu wollen.
Diese Benutzungsfälle, auch UseCases genannt, setzen sich aus verschiedenen Komponenten zusammen:
    \begin{itemize}
        \item Primary Actor\\
            Als "Primary Actor" wird der Hauptakteur des UseCases beschrieben.
            Der "Primary Actor" kann, muss aber nicht, eine natürliche Person darstellen. Als Beispiel kann ein "Primary Actor" auch
            ein Unternehmen darstellen. Die einzige einschränkung besteht darin, dass ein Akteur Entscheidungen treffen können muss.
            Der "Primary Actor" ist im gleichen zuge auch ein im folgenden Element beschriebener "Stakeholder". "Stakeholders" müssen
            aber keine "Primary Actors" sein.
        \item Stakeholder and Interests\\
            "Stakeholder and Interests" sitzen im Vergleich zum "Primary Actor" auf der passiven Seite. Sie greifen nicht direkt
            indas Geschehen ein, sind aber trotzdem an einem gewissen Ergebnis interessiert.
        \item Preconditions\\
            Als "Precondition" wird der initiale Zustand des UseCases beschrieben, welcher für eine Abarbeitung nötig ist.
        \item Postcondition\\
            In der "Postcondition" wird der Endzustand des Usecases beschrieben. Dieser enthält ausschließlich das optimale Ergebnis.
            Fehler oder Ausnahmen werden in den "Extensions" beschrieben und können den Endzustand beeinflussen.
        \item Main Success Scenario\
            Das "Main Success Scenario" beschreibt nun den genauen Ablauf, wie das in der "Postcondition" optimale Endergebnis erreicht werden kann.
            Dabei geht man chronologisch aufsteigend vor. "Primary Actor" und System wechseln sich hierbei ab.
        \item Extensions\\
            Unter den "Extensions" versteht man Erweiterungen des "Main Success Scenario". Dabei kann es sich um Fehler oder Außnahmen handeln,
            welche bei der Abarbeitung des "Main Success Scenario" entstehen können. Desweiteren können in den "Extensions" auch
            weitere Pfade definiert werden, welche anschließend wieder ins "Main Success Scenario"  zurückweisen.
    \end{itemize}
\end{document}