%%
%% Author: Oliver Heil
%% 08.03.2018
%%

\documentclass[./detailed_overview_usecases.tex]{subfiles}
\begin{document}

    \subsection{Reservierung Optionen}
    \subsubsection{Detaillierte Benutzungsfallbeschreibungen}
    \textit{Primary Actor:}
    \begin{enumerate}
        Front-Office Personal
        Back-Office Personal
    \end{enumerate}
    \\
    \textit{Stakeholder and Interests:}
    \begin{itemize}
        \item[-] Front/Back-Office Personal: Möchte für eine bestehende Reservierung Optionen anlegen, löschen oder bearbeiten.
        \item[-] Individualgast: Möchte seine Reservierungsoption anpassen lassen.
    \end{itemize}

    \subsubsection*{Preconditions}
    Es muss eine bestehende Reservierung vorhanden sein, in welcher die gewünschte Option abgeändert beziehungsweise gelöscht oder hinzugefügt werden darf.

    \subsubsection*{Postconditions}
    Die Reservierung hat je nach Fall eine Option mehr oder weniger beziehungsweise eine bestehende Option geändert.

    \subsubsection*{Main Success Scenario}
    \begin{enumerate}
        \item Das Front/Back-Office Personal tätigt die gewünschte Änderung (hinzufügen, löschen, ändern) an den Optionen in der Reservierung.
        \item Das System speichert die getätigten Änderungen ab.
    \end{enumerate}

    \subsubsection*{Extensions}
    -
    \subsubsection{Sequenz Diagramme}
    \subsubsection{Kontrakte}
\end{document}