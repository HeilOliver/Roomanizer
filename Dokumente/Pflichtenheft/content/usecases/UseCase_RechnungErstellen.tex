%%
%% Author: Robert
%% 08.03.2018
%%

\documentclass[./detailed_overview_usecases.tex]{subfiles}
\begin{document}

    \subsection{Rechnung erstellen}
    \subsubsection{Detaillierte Benutzungsfallbeschreibungen}
    \textit{Primary Actor: Front/Back-Office Personal}
    % describe actor here
    \\
    \textit{Stakeholder and Interests:}
    \begin{itemize}
        \item[-] Front/Back-Office Personal: Möchte dem Gast mitteilen was er zu zahlen hat.
        \item[-] Individualgast: Möchte eine Bestätigung über eine getätigte Zahlung.
    \end{itemize}

    \subsubsection*{Preconditions}
	Die Rechnung muss vom Front/Back-Office Personal dem Individualgast vorgelegt werden und von diesem Bestätigt sein.

    \subsubsection*{Postconditions}
    Der Individualgast hat eine Bestätigung seiner Zahlung, wenn diese nicht auf Kredit beglichen wurde.

    \subsubsection*{Main Success Scenario}
    \begin{enumerate}
        \item Das Front/Back-Office Personal geht mit dem Individualgast die Rechnung durch und tätigt allfällige Änderungen (Anpassung der Anschrift, Individualgastdaten).
		\item Der Individualgast teilt dem Front/Back-Office Personal seine gewünschte Zahlungsart mit. Das Front/Back-Office Personal trägt die Zahlungsart in das System ein und startet die Zahlung.
        \item Das System überprüft die getätigte Zahlung und liefert die erstellte Rechnung mit den gewünschten Informationen.
        \item Das Front/Back-Office Personal händigt dem Individualgast die erstellte Rechnung aus.
    \end{enumerate}

    \subsubsection*{Extensions}
    \begin{enumerate}
        \setcounter{enumi}{2}
        \item Kredit:
            \begin{itemize}
            \item[a.] Das System zeigt an, dass der Gast die Rechnung zugesendet haben möchte.
            \begin{itemize}
                \item[i.] Der Rechnungsstatus wird auf Kredit gesetzt.
                \item[ii.] Die Rechnung wird zugesendet.
            \end{itemize}
        \end{itemize}
    \end{enumerate}
    \subsubsection{Sequenz Diagramme}
    \subsubsection{Kontrakte}
\end{document}