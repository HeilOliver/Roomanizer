%%
%% Author: Robert
%% 08.03.2018
%%

\documentclass[./detailed_overview_usecases.tex]{subfiles}
\begin{document}

    \subsection{Rechnung stornieren}
    \subsubsection{Detaillierte Benutzungsfallbeschreibungen}
    \textit{Primary Actor: Front/Back-Office Personal}
    % describe actor here
    \\
    \textit{Stakeholder and Interests:}
    \begin{itemize}
        \item[-] Front/Back-Office Personal: Möchte die gelegte Rechnung stornieren.
        \item[-] Buchhaltung: Wird darüber informiert, dass eine Rechnung storniert wird.
    \end{itemize}

    \subsubsection*{Preconditions}
    Es muss eine Rechnung gelegt worden sein (UseCase: Rechnung legen).

    \subsubsection*{Postconditions}
    Die Rechnung ist storniert und es kann eine neue angelegt werden. Die Nummer der alten Rechnung bleibt bestehen.

    \subsubsection*{Main Success Scenario}
    \begin{enumerate}
        \item Das Front/Back-Office Personal oder der Individualgast finden einen Fehler in der Rechnung, beziehungsweise das Front/Back-Office Personal wird darüber informiert, dass die Rechnung nicht mehr gültig ist.
        \item Das System storniert die bereits gelegte Rechnung, löscht diese aber nicht.
        \item Der Buchhaltung wird mitgeteilt, dass diese Rechnung nicht mehr gültig ist.
    \end{enumerate}

    \subsubsection*{Extensions}
    \begin{enumerate}
        \setcounter{enumi}{3}
        \item Teilfehler:
        \begin{itemize}
            \item[a.] Auf der Rechnung ist nicht alles falsch.
            \begin{itemize}
                \item[i.] Es wird ein Teil der Rechnung storniert (UseCase: Rechnungsposition stornieren).
            \end{itemize}
        \end{itemize}
    \end{enumerate}
    \subsubsection{Sequenz Diagramme}
    \subsubsection{Kontrakte}
\end{document}