%%
%% Author: Robert
%% 08.03.2018
%%

\documentclass[./detailed_overview_usecases.tex]{subfiles}
\begin{document}

    \subsection{Zwischenrechnung erstellen}
    \subsubsection{Detaillierte Benutzungsfallbeschreibungen}
    \textit{Primary Actor: Front/Back-Office Personal}
    % describe actor here
    \\
    \textit{Stakeholder and Interests:}
    \begin{itemize}
        \item[-] Front/Back-Office Personal: Möchte eine Zwischenrechnung erstellen um einen Überblick über die Positionen zu erhalten.
        \item[-] Individualgast: Möchte einen Überblick über die zu zahlenden Positionen erhalten.
    \end{itemize}

    \subsubsection*{Preconditions}
    Es muss ein Aufenthalt vorhanden sein der dem Individualgast zugewiesen wurde.

    \subsubsection*{Postconditions}
    Die Rechnung kann gelegt werden (UseCase_RechnungLegen).

    \subsubsection*{Main Success Scenario}
    \begin{enumerate}
        \item Der Individualgast möchte seinen Aufenthalt beenden.
        \item Das Personal teilt dem System mit eine Zwischenrechnung zu erstellen mit den gewünschten Informationen.
        \item Das System gibt die Zwischenrechnung aus.
        \item Der Individualgast überprüft die Zwischenrechnung.
        \item Die Rechnung wird gelegt (UseCase_RechnungLegen).
    \end{enumerate}

    \subsubsection*{Extensions}
    \begin{enumerate}
        \setcounter{enumi}{3}
        \item Fehler:
        \begin{itemize}
            \item[a.] Dem Individualgast fällt ein Fehler auf, oder die Rechnung muss aufgeteilt werden.
            \begin{itemize}
                \item[i.] Die Zwischenrechnung wird verändert oder geteilt (UseCase_RechnungTeilen).
                \item[ii.] Der Individualgast bestätigt die neue Rechnung.
            \end{itemize}
        \end{itemize}
    \end{enumerate}

\end{document}