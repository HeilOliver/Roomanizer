%%
%% Author: Oliver Heil
%% 08.03.2018
%%


\documentclass[./detailed_overview_usecases.tex]{subfiles}
\begin{document}

    \subsection{Reservierung stornieren}
    \subsubsection{Detaillierte Benutzungsfallbeschreibungen}
    \textit{Primary Actor:}
    Front/Back-Office Personal
    \\
    Stakeholder and Interests:
    \begin{itemize}
        \item[-] Front/Back-Office Personal: Stornierte Reservierungen sollen gelöscht werden.
        \item[-] Gast: Möchte bei Reiseplanänderung von einer Reservierung zurücktreten.
    \end{itemize}

    \subsubsection*{Preconditions}
    Eine gültige Reservierung.

    \subsubsection*{Postconditions}
    Die Reservierung wurde systemintern storniert und belegt keine Hotel Ressourcen mehr.

    \subsubsection*{Main Success Scenario}
    \begin{enumerate}
        \item Der Gast wählt einen Stornierungsgrund aus.
        \item Der Gast bestätigt die Stornierung.
        \item Das System speichert die Stornierung.
        \item Der/Die Anwender/In zahlt bereits getätigte Anzahlungen aus.
    \end{enumerate}

    \subsubsection*{Extensions}
    \begin{enumerate}
        \item Der Gast möchte nach einer Stornierung die Reservierung trotzdem wahrnehmen.
        \begin{itemize}
            \item[a.] Das Front/Back-Office Personal muss eine neue Reservierung erstellt werden.
        \end{itemize}
        \item Die Reservierung ist außerhalb des Stornierungsfrist.
        \begin{itemize}
            \item[a.] Der Gast muss eine Stornierungsgebühr zahlen.
        \end{itemize}
    \end{enumerate}

\end{document}