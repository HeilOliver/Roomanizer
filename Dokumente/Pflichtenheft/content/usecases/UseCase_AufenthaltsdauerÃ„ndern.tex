%%
%% Author: Stefan Geiger
%% 13.03.2018
%%


\documentclass[./detailed_overview_usecases.tex]{subfiles}
\begin{document}

    \subsection{Aufenthaltsdauer ändern}
    \subsubsection{Detaillierte Benutzungsfallbeschreibungen}
    \textit{Primary Actor:}
    Front Office Mitarbeiter
	Back Office Mitarbeiter
	Gast/Kunde
    \\
    Stakeholder and Interests:
    \begin{itemize}
        \item[-] Front/Back Office Mitarbeiter: Aufenthaltsdauer einer Reservierung beziehungsweise Aufenthalt ändern.
        \item[-] Gast/Kunde: Möchte seine Aufenthalsdauer ändern.
    \end{itemize}

    \subsubsection*{Preconditions}
	Eine bestehende Reservierung beziehungsweise ein aktueller Aufenthalt muss vorhanden sein.
	
    \subsubsection*{Postconditions}
    Der Gast/Kunde konnte seinen Aufenthaltsdauer nach seinen Wünschen entsprechend ändern.
	
    \subsubsection*{Main Success Scenario}
    \begin{enumerate}
        \item Der Gast/Kunde gibt dem Front/Back Office Mitarbeiter seine gewünschte Änderung der Aufenthaltsdauer bekannt.
        \item Der Front/Back Office Mitarbeiter ruft die Reservierung beziehungsweise den Aufenthalt des Gastes/Kunde auf und gibt die Änderung ein.
	    \item Das System prüft diese Änderung und gibt anschließend eine Bestätigung aus, welche der Front/Back Office Mitarbeiter bestätigen muss.
	    \item Der Front/Back Office Mitarbeiter teilt dem Gast/Kunde die erfolgreiche Änderung mit und bestätigt diese.
    \end{enumerate}

    \subsubsection*{Extensions}
    \begin{enumerate}
        \item 
		\item 
		\item Im neuen Zeitraum ist kein Zimmer frei beziehungsweise ein Zimmerwechsel erforderlich.
        \begin{itemize}
                       \item[a.] Der Front/Back Office Mitarbeiter teilt dem Gast/Kunde mit, dass kein Zimmer im gewünschten Zeitraum frei ist aber in einer höheren Kategorie.
					   \item[a2.] Der Gast/Kunde akzeptiert das Zimmer aus der höheren Kategorie, Sprung zu Use-Case Zimmewechsel
					   \item[a3.] Der Gast/Kunde akzeptiert das Zimmer nicht und verkürzt seinen Aufenthalt.
                       
        \end{itemize}
    \end{enumerate}

    \subsubsection{Sequenz Diagramme}
    \subsubsection{Kontrakte}
\end{document}