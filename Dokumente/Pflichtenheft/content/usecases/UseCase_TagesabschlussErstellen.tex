%%
%% Author: Robert
%% 08.03.2018
%%

\documentclass[./detailed_overview_usecases.tex]{subfiles}
\begin{document}

    \subsection{Tagesabschluss erstellen}
    \subsubsection{Detaillierte Benutzungsfallbeschreibungen}
    \textit{Primary Actor: Back-Office Personal}
    % describe actor here
    \\
    \textit{Stakeholder and Interests:}
    \begin{itemize}
        \item[-] Back-Office Personal: Einfache Abwicklung des Abschlusses und korrekte Übersicht der Daten.
        \item[-] Hotelmanager: Korrekte Übersicht der Daten.
        \item[-] Buchhaltung: Korrekte Übersicht der Daten.
    \end{itemize}

    \subsubsection*{Preconditions}
    Daten müssen dem Tagesabschluss zugeordnet worden sein.
    \subsubsection*{Postconditions}
    Der Tagesabschlussbericht ist gedruckt und die Preise und Zusatzleistungen sind auf die jeweiligen Rechnungen gebucht.

    \subsubsection*{Main Success Scenario}
    \begin{enumerate}
        \item Das Back-Office Personal möchte einen Tagesabschluss erstellen.
        \item Das System rechnet die Kassabewegungen ab (UseCase KassaabschlussErstellen).
        \item Das System liefert dem Back-Office Personal den gewünschten Bericht.
        \item Das System ändert den Zimmerstatus der belegten Zimmer auf UNGEREINIGT (UseCase ZimmerstatusÄndern).
        \item Das Back-Office Personal druckt den Bericht aus und übergibt den Bericht an die Buchaltung.
    \end{enumerate}

    \subsubsection*{Extensions}

    \subsubsection{Sequenz Diagramme}
    \subsubsection{Kontrakte}
\end{document}