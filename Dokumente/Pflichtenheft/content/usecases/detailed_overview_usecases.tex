%%
%% Author: Moritz
%% 08.03.2018
%%

% Preamble
\documentclass[../../Pflichtenheft.tex]{subfiles}
\begin{document}
    %\subfile{./content/usecases/usecase_template.tex}
    % add detailed use cases here via subfile
    Für alle Use Cases gelten folgende Preconditions:
    \begin{itemize}
        \item[*] \hspace{0.4cm}
            \begin{itemize}
                \item Der Benutzer des Systems ist im System authentifiziert und hat die Berechtigung für die aufgezählten Systemaktionen.
            \end{itemize}
    \end{itemize}
    Für alle Extensions gilt folgende Möglichkeit:
    \begin{itemize}
        \item[*] \hspace{0.4cm}
        \begin{itemize}
            \item Ein UseCase kann jeder Zeit abgebrochen werden. Für diesen Fall werden vom System keinerlei Daten gespeichert.
        \end{itemize}
    \end{itemize}

    \subfile{./content/usecases/UseCase_GastKundeAnlegen.tex}
    \subfile{./content/usecases/UseCase_ZimmerstatusSetzen.tex}
    \subfile{./content/usecases/UseCase_ReservierungIndividualgast.tex}
    \subfile{./content/usecases/UseCase_StammdatenÄndern.tex}
    \subfile{./content/usecases/UseCase_ReservierungStonieren.tex}
    \subfile{./content/usecases/UseCase_Zimmerzuteilung.tex}
    \subfile{./content/usecases/UseCase_AufenthaltsdauerÄndern.tex}
    \subfile{./content/usecases/UseCase_AkontoBuchen.tex}
    \subfile{./content/usecases/UseCase_CheckInMitReservierung.tex}
    \subfile{./content/usecases/UseCase_Checkout.tex}
    \subfile{./content/usecases/UseCase_ExtraleistungenBuchen.tex}
    \subfile{./content/usecases/UseCase_GastKundeAnlegen.tex}
    \subfile{./content/usecases/UseCase_JahresabschlussErstellen.tex}
    \subfile{./content/usecases/UseCase_Kassastonierung.tex}
    \subfile{./content/usecases/UseCase_KassenAbschluss.tex}
    \subfile{./content/usecases/UseCase_MonatsabschlussErstellen.tex}
    \subfile{./content/usecases/UseCase_RechnungErstellen.tex}
    \subfile{./content/usecases/UseCase_TagesabschlussErstellen.tex}
    \subfile{./content/usecases/UseCase_ZimmerWechseln.tex}

\end{document}