%%
%% Author: Oliver Heil
%% 08.03.2018
%%


\documentclass[./detailed_overview_usecases.tex]{subfiles}
\begin{document}

    \subsection{Individualgast Anlegen}
    \subsubsection{Detaillierte Benutzungsfallbeschreibungen}
    \textbf{Primary Actor:}
    \begin{itemize}
        \item[-] Front-Office Personal
        \item[-] Back-Office Personal
    \end{itemize}
    \\
    \textbf{Stakeholder and Interests:}
    \begin{itemize}
        \item[-] Front/Back-Office Personal: Anlegen von neuen Gästen/Kunden in der Gästekartei.
        \item[-] Individualgast: möchte in der Gästekartei sein um schneller Reservierungen oder Check-Ins/Check-Outs vorzunehmen
        und um gegebenenfalls über Neuigkeiten informiert zu werden (Beispiel Newsletter).
        \item[-] Hotelmanager: möchte alle Gästedaten gesammelt haben.
    \end{itemize}

    \subsubsection*{Preconditions}
    -

    \subsubsection*{Postconditions}
    Ein neuer Individualgast ist in der Gästekartei eingetragen.

    \subsubsection*{Main Success Scenario}
    \begin{enumerate}
        \item Das Front/Back-Office Personal trägt in das Formular alle Daten des neuen Individualgastes ein.
        \item Das System speichert die eingetragenen Daten in der Gästekartei.
    \end{enumerate}

    \subsubsection*{Extensions}
    \begin{enumerate}
        \item
        \begin{itemize}
            \item[a.] Die eingegebenen Daten sind nicht gültig beziehungsweise lückenhaft.
                \begin{itemize}
                       \item[i.] Die Daten werden nicht gespeichert.
                       \item[ii.] Das Front/Back-Office Personal kann die bestehenden Daten korrigieren.
                \end{itemize}
        \end{itemize}
        \item
        \begin{itemize}
            \item[a.] Der neu anzulegende Individualgast ist bereits in der Gästekartei vorhanden.
                \begin{itemize}
                        \item[i.] Die Daten werden nicht gespeichert.
                        \item[ii.] Das Front/Back-Office Personal kann die eingegebenen Daten korrigieren.
                \end{itemize}
        \end{itemize}
    \end{enumerate}

\end{document}