%%
%% Author: Oliver Heil
%% 08.03.2018
%%


\documentclass[./detailed_overview_usecases.tex]{subfiles}
\begin{document}

    \subsection{Reservierung Stornieren}
    \subsubsection{Detaillierte Benutzungsfallbeschreibungen}
    \textit{Primary Actor:}
    Front/Back office Personal
    \\
    Stakeholder and Interests:
    \begin{itemize}
        \item[-] Front/Back office Personal: Nicht mehr gültige Reservierungen müssen gelöscht werden um Ressourcen freizugeben
        \item[-] Gast/Kunde: Möchte bei Reiseplan änderung von Reservierung zurücktreten
    \end{itemize}

    \subsubsection*{Preconditions}
    Eine Person des Front/Back offices die die benötigte Berechtigungsstufe aufweist um eine änderung an einer Reservierung vorzunehmen.

    \subsubsection*{Postconditions}
    Die Reservierung wurde storniert und belegt keine Hotel ressourcen mehr

    \subsubsection*{Main Success Scenario}
    \begin{enumerate}
        \item Der/Die Anwender/In wählt einen stornierung Grund aus.
        \item Der/Die Anwender/In Bestätigt die stornierung.
        \item Das System speichert die stornierung und gibt die durch die Reservierung belegten ressourcen frei.
    \end{enumerate}

    \subsubsection*{Extensions}
    \begin{enumerate}
        \item Der Gast möchte nach einer Stornierung die Reservierung trotzdem wahrnehmen
        \begin{itemize}
            \item[a.] Es muss eine neue Reservierung erstellt werden.
        \end{itemize}
    \end{enumerate}

    \subsubsection{Sequenz Diagramme}
    \subsubsection{Kontrakte}
\end{document}