%%
%% Author: Oliver Heil
%% 08.03.2018
%%


\documentclass[./detailed_overview_usecases.tex]{subfiles}
\begin{document}

    \subsection{Zimmerzuteilung}
    \subsubsection{Detaillierte Benutzungsfallbeschreibungen}
    \textit{Primary Actor:}
    \begin{enumerate}
        Front-Office Personal
        Back-Office Personal
    \end{enumerate}
    \\
    Stakeholder and Interests:
    \begin{itemize}
        \item[-] Front/Back-Office Personal: Kann das Zimmer beim Check-In vom Kunden bezogen werden.
        \item[-] Reinigungsfachkraft: Welche Zimmer zuerst gereinigt werden müssen.
        \item[-] Individualgast: Ein Zimmer in der richtigen Kategorie beziehungsweise das explizit gewünschte Zimmer.
    \end{itemize}

    \subsubsection*{Preconditions}
    -
    \subsubsection*{Postconditions}
    Alle Individualgäste haben ein Zimmer beziehungsweise ihr explizit gewünschtes Zimmer im angegebenen Zeitraum zugeteilt bekommen.

    \subsubsection*{Main Success Scenario}
    \begin{enumerate}
        \item Der/Die Anwender/In wählt die Zimmerzuteilung mit bestimmten Ankunftstag.
        \item Das System zeigt alle am ausgewählten Ankunftstag entsprechenden fixen Reservierungen an.
        \item Der/Die Anwender/In wählt eine Reservierung aus.
        \item Das System zeigt die reservierten Positionen (Anzahl, Kategorie) der Reservierung an.
        \item Der/Die Anwender/In wählt eine Position aus.
        \item Das System zeigt alle Zimmer an, welche zuteilbar sind.
        \item Der/Die Anwender/In weist ein Zimmer zu.
        \item Das System zeigt die aktualisierte Zimmerzuteilung an.
        \item Der/Die Anwender/In bestätigt diese Zimmerzuteilung.
        \item Das System speichert diese Zimmerzuteilung ab.
        \item Der/Die Anwender/In wiederholt die Schritte 4-10 beziehungsweise 1-10, bis dieser Use-Case beendet ist.
    \end{enumerate}

    \subsubsection*{Extensions}
    \item 1-9 \begin{itemize}
                   \item[a.] Der/Die Anwender/In bricht diesen Use-Case ab.
    \end{itemize}
    \item 2 \begin{itemize}
                \item[a.] Es gibt keine fixen Reservierungen an diesem ausgewählten Ankunftstag.
                \begin{itemize}
                    \item[i.] Use-Case beenden
                \end{itemize}
    \end{itemize}
    \item 4 \begin{itemize}
                \item[a.] Das Zimmer wurde bei der Reservierung bereits zugeteilt.
    \end{itemize}
    \item 5 \begin{itemize}
                \item[a.] Der/Die Anwender/In lässt das System die Zimmer automatisch zuweisen
                \begin{itemize}
                    \item[i.] Springe zu Schritt 8
                \end{itemize}
    \end{itemize}
    \item 6 \begin{itemize}
                \item[a.] Es gibt kein freies Zimmer im gewünschten Zeitraum.
                \begin{itemize}
                    \item[i.] Der/Die Anwender/In sucht ein Zimmer mit einem anderen Zeitraum
                \end{itemize}
                \item[b.] Es gibt kein freies Zimmer in der gewünschten Kategorie
                \begin{itemize}
                    \item[i.] Der/Die Anwender/In wählt ein Zimmer in einer höheren Kategorie.
                \end{itemize}
                \item[c.] Es ist kein Zimmer verfügbar.
                \begin{itemize}
                    \item[i.] Der/Die Anwender/In versucht eine Lösung mit dem Gast zu erzielen.
                \end{itemize}
    \end{itemize}
    \item 8 \begin{itemize}
                \item[a.] Es sind mehrere Zimmer je Position verbucht.
                \begin{itemize}
                    \item[i.] Der Mitarbeiter trägt einen Zimmerwechsel im System ein
                \end{itemize}
    \end{itemize}
    \item 9 \begin{itemize}
                \item[a.] Der Mitarbeiter möchte die Zimmerzuteilung ändern.
                \begin{itemize}
                    \item[i.] Springe zu Schritt 5
                \end{itemize}
    \end{itemize}
    \subsubsection{Sequenz Diagramme}
    \subsubsection{Kontrakte}
\end{document}