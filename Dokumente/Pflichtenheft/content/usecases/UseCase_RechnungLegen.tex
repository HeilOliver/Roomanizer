%%
%% Author: Robert
%% 08.03.2018
%%

\documentclass[./detailed_overview_usecases.tex]{subfiles}
\begin{document}

    \subsection{Rechnung legen}
    \subsubsection{Detaillierte Benutzungsfallbeschreibungen}
    \textit{Primary Actor: Front/Back-Office Personal}
    % describe actor here
    \\
    \textit{Stakeholder and Interests:}
    \begin{itemize}
        \item[-] Front/Back-Office Personal: Möchte die Rechnung im System fixieren.
        \item[-] Buchhaltung: Möchte die erstellte Rechnung erhalten.
    \end{itemize}

    \subsubsection*{Preconditions}
    Es muss eine Zwischenrechnung erstellt (UseCase_ZwischenrechnungErstellen) sein.

    \subsubsection*{Postconditions}
    %post conditions text
    Die Rechnung kann bezahlt werden und die Buchhaltung hat die Rechnung erhalten.

    \subsubsection*{Main Success Scenario}
    \begin{enumerate}
        \item Das Personal teilt dem System mit die Rechnung zu legen.
        \item Das System legt die Rechnung und sendet diese an die Buchhaltung.
        \item Der Individualgast überprüft die Rechnun.
        \item Das Personal bestätigt die Legung
    \end{enumerate}

    \subsubsection*{Extensions}
    \begin{enumerate}
        \setcounter{enumi}{3}
        \item Fehler:
        \begin{itemize}
            \item[a.] Dem Individualgast fällt ein Fehler auf.
            \begin{itemize}
                \item[i.] Die gelegte Rechnung muss storniert werden (UseCase_RechnungStornieren).
            \end{itemize}
        \end{itemize}
    \end{enumerate}
    \subsubsection{Sequenz Diagramme}
    \subsubsection{Kontrakte}
\end{document}