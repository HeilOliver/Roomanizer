%%
%% Author: Robert
%% 08.03.2018
%%

\documentclass[./detailed_overview_usecases.tex]{subfiles}
\begin{document}

    \subsection{Reservierung Individualgast}
    \subsubsection{Detaillierte Benutzungsfallbeschreibungen}
    \textit{Primary Actor: Front-Office Mitarbeiter}
    % describe actor here
    \\
    \textit{Stakeholder and Interests:}
    \begin{itemize}
        \item[-] Rezeptionist: schnelle, fehlerfreie und einfache Abwicklung der Reservierung, Zimmerzuteilung, Zusatzleistungen buchen, Änderungen an bestehenden Reservierungen vornehmen und Möglichkeit Zwischenrechnung sowie Reservierungsbestätigung zu drucken.
        \item[-] Individualgast: Reservierung eines Zimmers ohne Komplikationen hinsichtlich seines Aufenthaltes, Möglichkeit Zusatzleistungen zu buchen. Möchte eventuell Zwischenrechnung und Reservierungsbestätigung.
        \item[-] Hotelmanager: Möchte ebenfalls, dass der Rezeptionist im Stande ist Reservierung schnell und fehlerfrei abzuwickeln, sodass der Kunde zufrieden ist. Möchte alle Statistiken in Zusammenhang mit der Reservierung abrufen.
    \end{itemize}

    \subsubsection*{Preconditions}
    Informationen darüber ob der Individualgast bereits Kunde des Hotels war bzw. ob es sich um einen Gast des Hauses handelt.

    \subsubsection*{Postconditions}
    %post conditions text
    Zimmer ist für einen bestimmten Zeitraum auf den Individualgast reserviert

    \subsubsection*{Main Success Scenario}
    \begin{enumerate}
        \item Der Individualgast nennt den gewünschten Reservierungszeitraum und die Art des Zimmers (Kategorie, WLAN, Haustiere usw.)
        \item Der Rezeptionist gibt den vom Individualgast erhaltenen Zeitraum und die gewünschten Präferenzen in das System ein.
        \item Das System liefert dem Rezeptionisten die gewünschten Informationen ob und welche Zimmer in welcher Kategorie frei sind.
        \item Der Individualgast bestätigt, dass er eines dieser Zimmer zum gewünschten Zeitraum belegen möchte und keine weiteren Zimmer reservieren möchte.
        \item Die Zusatzpakete werden gebucht (UseCase: Zusatzpakete bestellen).
        \item Der Individualgast wird im System angelegt (UseCase: Kunde/Gast anlegen).
        \item Der Individualgast überprüft die eingegebenen Daten und bestätigt die Reservierung.
        \item Der Rezeptionist schließt die Reservierung ab und druckt eine Reservierungsbestätigung für den Gast.
    \end{enumerate}

    \subsubsection*{Extensions}
    \begin{enumerate}
        \setcounter{enumi}{2}
        \item Überbuchung:
                \begin{itemize}
                       \item[a.] Das System zeigt an, dass kein Zimmer mehr verfügbar ist und der Rezeptionist hat die Berechtigung zu überbuchen.
                            \begin{itemize}
                                \item[i.] Die Reservierung wird an dieser Stelle fortgesetzt
                                \item[ii.] Punkt 4 des Main Success Szenarios wird aufgerufen
                            \end{itemize}
                       \item[b.] Das System zeigt an, dass kein Zimmer mehr verfügbar ist und der Rezeptionist hat nicht die Berechtigung zu überbuchen.
                            \begin{itemize}
                                \item[i.] Der Individualgast kann einen anderen Zeitraum oder andere Präferenzen auswählen.
                                \item[ii.] Punkt 2 des Main Success Szenarios wird aufgerufen
                            \end{itemize}
                       \item[c.] Das System zeigt an, dass keine Überbuchungen mehr möglich sind.
                            \begin{itemize}
                                \item[i.] Der Individualgast kann einen anderen Zeitraum oder andere Präferenzen auswählen.
                                \item[ii.] Punkt 2 des Main Success Szenarios wird aufgerufen
                             \end{itemize}
                \end{itemize}
        \setcounter{emuni}{4}
        \item Preis festlegen
            \begin{itemize}
                \item[a.] Der Gast ist Gast des Hauses
                \begin{itemize}
                    \item[i.] Der Preis wird vom Rezeptionisten auf 0 gesetzt und die Reservierung auf "Gast des Hauses" gesetzt.
                    \item[ii.] Punkt 5 des Main Success Szenarios wird aufgerufen
                \end{itemize}
            \end{itemize}
        \item Preis überprüfen
        \begin{itemize}
            \item[a.] Der eingebene Preis ist ungültig
            \begin{itemize}
                \item[i.] Das System signalisiert dem Front-Office Mitarbeiter, dass ein Fehler vorliegt und was der Grund dafür ist
                \item[ii.] Der Front-Office Mitarbeiter gibt den Preis erneut ein bis das System die eingebenen Daten annimmt.
                \item[iii.] Punkt 5 des Main Success Szenarios wird aufgerufen
            \end{itemize}
        \end{itemize}
        \setcounter{enumi}{8}
        \item Unvollständige Daten: \begin{itemize}
                                        \item[a.] Der Individualgast bestätigt die vorliegende Reservierung nicht, da bestimmte Daten fehlen oder nicht korrekt sind.
                                        \begin{itemize}
                                            \item[i.] Die Reservierung wird an der Stelle neu gestartet an dieser der Fehler aufgetreten ist.
                                        \end{itemize}
                                    \end{itemize}
    \end{enumerate}

    \subsubsection{Sequenz Diagramme}
    \subsubsection{Kontrakte}
\end{document}