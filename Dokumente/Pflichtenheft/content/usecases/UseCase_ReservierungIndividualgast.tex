%%
%% Author: Moritz
%% 08.03.2018
%%

\documentclass[./detailed_overview_usecases.tex]{subfiles}
\begin{document}

    \subsection{Reservierung Individualgast}


    \textit{Primary Actor: Rezeptionist}
    % describe actor here
    \newline
    Stakeholder and Interests:
    \begin{itemize}
        \item[-] Rezeptionist: schnelle und fehlerfreie Abwicklung der Reservierung, neuen Kunden anlegen, einfach Handhabung der Software.
        \item[-] Individualgast: Reservierung eines Zimmers ohne Komplikationen hinsichtlich seines Aufenthaltes, Möglichkeit Zusatzleistungen zu buchen.
        \item[-] Hotelmanager: Möchte ebenfalls, dass der Rezeptionist im Stande ist Reservierung schnell und fehlerfrei abzuwickeln, sodass der Kunde zufrieden ist.
    \end{itemize}

    \subsubsection*{Preconditions}
    Informationen darüber ob der Individualgast bereits Kunde des Hotels war bzw. ob es sich um einen Gast des Hauses handelt.

    \subsubsection*{Postconditions}
    %post conditions text
    Zimmer ist für einen bestimmten Zeitraum auf den Individualgast reserviert

    \subsubsection*{Main Success Scenario}
    \begin{enumerate}
        \item Der Individualgast nennt den gewünschten Reservierungszeitraum und die Art des Zimmers (Kategorie,WLAN,Haustiere usw.)
        \item Der Rezeptionist gibt den vom Individualgast erhaltenen Zeitraum und die gewünschten Präferenzen in das System ein.
        \item Das System liefert dem Rezeptionist die gewünschten Informationen ob und welche Zimmer frei sind.
        \item Der Individualgast bestätigt, dass er eines dieser Zimmer zum gewünschten Zeitraum belegen möchte.
        \item Die Zusatzpakete werden gebucht (UseCase_Zusatzpakete bestellen).
        \item Der Individualgast wird im System angelegt (UseCase_Kunde/Gast anlegen).
        \item Der Individualgast überprüft die eingegebenen Daten und bestätigt die Reservierung.
        \item Der Rezeptionist schließt die Reservierung ab.
    \end{enumerate}

    \subsubsection*{Extensions}
    \begin{enumerate}
        \setcounter{enumi}{3}
        \item Überbuchung: \begin{itemize}
                       \item[a.] Das System zeigt an, dass kein Zimmer mehr verfügbar ist und der Rezeptionist hat die Berechtigung zu überbuchen.
                       \begin{itemize}
                           \item[i.] Die Reservierung wird an dieser Stelle fortgesetzt
                           \item[ii.] Punkt 9 des Main Success Szenarios wird aufgerufen
                       \end{itemize}
                       \item[b.] Das System zeigt an, dass kein Zimmer mehr verfügbar ist und der Rezeptionist hat nicht die Berechtigung zu überbuchen.
                       \begin{itemize}
                           \item[i.] Der Individualgast kann einen anderen Zeitraum oder andere Präferenzen auswählen.
                           \item[ii.] Punkt 2 des Main Success Szenarios wird aufgerufen
                       \end{itemize}
                       \item[c.] Das System zeigt an, dass keine Überbuchungen mehr möglich sind.
                       \begin{itemize}
                           \item[i.] Der Individualgast kann einen anderen Zeitraum oder andere Präferenzen auswählen.
                           \item[ii.] Punkt 2 des Main Success Szenarios wird aufgerufen
                       \end{itemize}
        \end{itemize}
        \setcounter{enumi}{7}
        \item Unvollständige Daten: \begin{itemize}
                               \item[a.] Der Individualgast bestätigt die vorliegende Reservierung nicht, da bestimmte daten fehlen oder nicht korrekt sind.
                               \begin{itemize}
                                   \item[i.] Die Reservierung wird an der Stelle neu gestartet an dieser der Fehler aufgetreten ist.
                               \end{itemize}
        \end{itemize}
    \end{enumerate}

\end{document}