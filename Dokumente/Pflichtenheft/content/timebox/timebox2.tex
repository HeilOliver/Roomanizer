%%
%% Author: Moritz
%% 16.03.2018
%%

% Preamble
\documentclass[../Pflichtenheft.tex]{subfiles}
\begin{document}
    \subsection{Benutzungsfall/fälle (UseCase(s)) }
    Für die zweite Timebox sollen folgende UseCases implementiert:
    \begin{itemize}
        \item \textbf{Check-In:} Alle Punkte des Main Success Scenarios werden umgesetzt.
        Alle Extensions sollen berücksichtigt werden.
        \item \textbf{Zimmerstatus ändern: } Alles aus diesem UseCase wird realisiert.
        \item \textbf{Walk-In: } In Zusammenhang mit der Reservierung und dem Check-In wird dieser Use Case vollständig realisiert.
        \item \textbf{Zimmerzuteilung: } Der komplette Use Case 'Zimmerzuteilung' soll in Timebox 2 implementiert werden.
        \item \textbf{Check-Out:} Die verknüpften Use Cases zum Rechnungswesen werden nicht berücksichtigt. Das dient dann als Vorbereitung für die dritte Timebox.
        \item \textbf{Rechnung erstellen:} Komplette Umsetzung des Use Cases
        \item \textbf{Rechnung legen:} Komplette Umsetzung des Use Cases
        \item \textbf{AKonto buchen: } Der UseCase 'Akonto buchen' wird vollständig implementiert.
    \end{itemize}
    \subsection{Architektur}
    \begin{itemize}
        \item Erweiterung der JavaFX GUI
    \end{itemize}
    \subsection{Deliverables}
    Folgende Dokumente sollen am Ende der Timebox verfügbar sein: \\
    Code:
    \begin{itemize}
        \item Java 9 Source Code
        \item JavaDoc
        \item Basis für Timebox 3 (Rechnungswesen)
    \end{itemize}
    Unit- und Integrationtests:
    \begin{itemize}
        \item JUnit-Tests inklusive der Testdaten
    \end{itemize}
    \subsection{Abhängigkeiten}
    Folgende Vorraussetzungen für die Erfüllung der Timebox werden benötigt:
    \begin{itemize}
        \item Grundlegende Architektur
        \item Erfolgreiche Umsetzung der Use Cases aus Timebox 1
    \end{itemize}
\end{document}