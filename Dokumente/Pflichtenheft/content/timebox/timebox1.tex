%%
%% Author: Moritz
%% 16.03.2018
%%

% Preamble
\documentclass[../Pflichtenheft.tex]{subfiles}
\begin{document}
    \subsection{Benutzungsfall/fälle (UseCase(s)) }
        Für die erste Timebox sollen folgende UseCases implementiert:
        \begin{itemize}
            \item \textbf{Reservierung Individualgast: } Das gesamte Main Success Scenario ohne Punkt
            7 (Zusatzpakete) und Punkt 6 (Preis überprüfen) wird umgesetzt. Die Extensions 'Überbuchung'
            sowie die Reservierung für Vertragspartner sollen nicht berücksichtigt werden.
            \item \textbf{Individualgast anlegen: } Abhängig vom Use Case 'Reservierung Individualgast' wird
            auch das Anlegen neuer Gästedaten umgesetzt.
        \end{itemize}
    \subsection{Architektur}
    \begin{itemize}
        \item GUI mit JavaFX
        \item Vollständige Datenbankmodellierung mit Entity Relationship Modell
        \item Datenbankanbindung mit Hibernate
    \end{itemize}
    \subsection{Deliverables}
    Folgende Dokumente sollen am Ende der Timebox verfügbar sein: \\
    Code:
    \begin{itemize}
        \item Java 9 Source Code
        \item JavaDoc
        \item Grundlegende Architektur als Basis für weitere Time Boxes
    \end{itemize}
    Unit- und Integrationtests:
    \begin{itemize}
        \item JUnit-Tests inklusive der Testdaten
    \end{itemize}
    \subsection{Abhängigkeiten}
    Folgende Vorraussetzungen für die Erfüllung der Timebox werden benötigt:
    \begin{itemize}
        \item Daten über Gäste, Hotel und Zimmer
        \item Vollständige Definition aller Use Cases
        \item Vollständig konfigurierter Datenbankserver
        \item Technisch stabile und sinnvolle Architektur ist definiert
    \end{itemize}
\end{document}