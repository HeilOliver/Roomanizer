%%
%% Author: Moritz
%% 15.03.2018
%%

% Preamble
\documentclass[../Pflichtenheft.tex]{subfiles}
\begin{document}
    Die Arbeitsumgebung setzt sich aus einem Arbeits PC, bestehend aus PC, Bildschirm, Tastatur
    und Maus und einem Drucker zusammen und kann vom Nutzerunternehmen zu Verfügung gestellt werden.
    Die in diesem Dokument beschriebene Anwendung benötigt zur Ausführung das JAVA - Runtime (min 1.9), ein
    Datenbanksystem für die Persistierung von strukturierten Daten und ein Dateisystem für die Speicherung von
    weniger strukturierten Daten (Logos, Berichte, usw.).
    Die Anwendung ist als Fat-Client konzipiert. Bei einem Fat-Client liegt sowohl die Funktionalität als auch die Anwendungslogik beim Client selbst.
    Die verschiedenen Berechtigungsstufen werden durch ein Log-in in der Anwendung bewerkstelligt. Ungewollte Änderungen
    durch Benutzer ohne Berechtigung sind somit ausgeschlossen.
    Die Anwendung wurde für schnelle und unkomplizierte Arbeitsabläufe konzipiert und ist in vielerlei Hinsicht selbsterklärend.
    Trotzdem sollte dem Benutzer die gängigen Arbeitsabläufe bekannt sein.
    Für den Austausch von Daten mit bestehenden Systemen sind verschiedene Schnittstelen vorgesehen:
    \begin{itemize}
        \item Finanzbuchhaltung
        \item Debitorenbuchhaltung
        \item Food and Beverage Verwaltung
    \end{itemize}
\end{document}